\documentclass[xcolor=dvipsnames]{beamer}
\usecolortheme[named=Blue]{structure}
\usetheme{Madrid}
\setbeamertemplate{items}[rectangle]
\setbeamertemplate{blocks}[rounded][shadow=false]
\setbeamertemplate{navigation symbols}{}
\newtheorem{proposition}{Proposition}
\usepackage{graphicx}
\usepackage{amsmath}

\usepackage{times}
\usepackage{graphicx}
\usepackage{placeins}
\usepackage{float}
\usepackage{subfigure}
\usepackage{enumerate}
\usepackage{times}
\usepackage{bm}
\usepackage{amsmath}
\usepackage{upquote}
\usepackage{times}




\title [Vanderbilt University] {Trade Policy In Regulated Economies}
\author[Bernes Karacay]{\textbf{}\\
\vskip 0.1in (Presented by: Bernes Karacay)\\
\vskip 0.2in
Department of  Economics\\
  Vanderbilt University}
\institute[]{}
\date{}
\begin{document}

\begin{frame}
\titlepage
\end{frame}

\begin{frame}
\frametitle{Introduction}
\begin{itemize}
  \item Jagduish Bhagwati, described Foreign Direct Investment (FDI)as the investments that is made
  by Multi National Companies (MNC) in order to skip the protectionist measures that is taken by
  home country.
 \item Many countries use protectionist barriers to prevent job losses and protect home
 industry from computation which is stemmed from cheap imported goods. Companies can get
 rid of this protectionist barrier by opening a new branch in home country.
 \item Hence, tariff rates
 couldn't serve as a protection alone to domestic entrepreneur(either it is private or State Own
 Enterprise), we should also consider that restrict foreign direct investment directly.
\end{itemize}
\end{frame}

\begin{frame}
	\frametitle{State Own Enterprises}
	\begin{figure}
		\begin{center}
			\includegraphics[scale=0.85]{Soe.png}
		\end{center}
	\end{figure}
\end{frame}

\begin{frame}
	\frametitle{State Own Enterprises}
	\begin{figure}
		\begin{center}
			\includegraphics[scale=0.5]{Soe_History.png}
		\end{center}
	\end{figure}
\end{frame}

\begin{frame}
	\frametitle{Fdi Restriction Index}
	\begin{figure}
		\begin{center}
			\includegraphics[scale=0.35]{FdiRes.png}
		\end{center}
	\end{figure}
\end{frame}

\begin{frame}
	\frametitle{ Fdi Restriction Index}
	\begin{figure}
		\begin{center}
			\includegraphics[scale=0.6]{Res1.png}
		\end{center}
	\end{figure}
\end{frame}

\begin{frame}
	\frametitle{ Fdi Restriction Index}
	\begin{figure}
		\begin{center}
			\includegraphics[scale=0.6]{Res2.png}
		\end{center}
	\end{figure}
\end{frame}

\begin{frame}
	\frametitle{Weighted Tariff Rate}
	\begin{figure}
		\begin{center}
			\includegraphics[scale=0.75]{tariff.png}
		\end{center}
	\end{figure}
\end{frame}

\begin{frame}
	\frametitle{FDI Inflow}
	\begin{figure}
		\begin{center}
			\includegraphics[scale=0.2]{EnergyMining.png}
		\end{center}
	\end{figure}
\end{frame}

\begin{frame}
	\frametitle{FDI inflow}
	\begin{figure}
		\begin{center}
			\includegraphics[scale=0.2]{Manufacturing.png}
		\end{center}
	\end{figure}
\end{frame}

\begin{frame}
	\frametitle{FDI inflow}
	\begin{figure}
		\begin{center}
			\includegraphics[scale=0.2]{FD.png}
		\end{center}
	\end{figure}
\end{frame}

\begin{frame}
	\frametitle{Model}
The utility function is represented as
\begin{equation}
U=x_0+\frac{\theta}{\theta-1}x^{\frac{\theta-1}{\theta}}, \hspace{1cm} \theta>0, \theta\neq0
\end{equation}
where $ x_0 $ denotes consumption of the numeraire good and x is an index of consumption of the differentiated products and $\theta>0$ denotes the elasticity of demand for the CES aggregate. The consumption index is denoted as
\begin{equation}\label{eq:ces}
x=\bigg[\sum_{j \in N_d}x(j)^\frac{\epsilon-1}{\epsilon}+\sum_{j \in N_i}x(j)^\frac{\epsilon-1}{\epsilon}+\sum_{j \in N_m}x(j)^\frac{\epsilon-1}{\epsilon}+\sum_{j \in N_{soe}}x(j)^\frac{\epsilon-1}{\epsilon}\bigg]^\frac{\epsilon}{\epsilon-1}
, \hspace{1cm} \epsilon>1
\end{equation}
where $ x(j) $ represents of the product $ j $, and  $ N_h$, $N_i $,  $N_m $ are differentiated products from different sources: home, import , multinational companies produce at home; respectively. In addition, the $\epsilon>1$ is the own-price elasticity of demand of each brand. We denote by $p_j$ the prices of each good $j=d,i,m,soe$. 
\end{frame}

\begin{frame}
	\frametitle{Model}
Hence maximizing the utility subject to budget constraint gives us
\begin{equation} \label{eq:demand}
x_j=p_j^{-\epsilon}q^{\epsilon-\theta}, \hspace{0.5cm}for  \hspace{0.3cm} j=i,d,m\hspace{0.5cm} and \hspace{0.5cm} x=q^{-\theta}
\end{equation}
The cost of production good differ by the type of company: $c_j$, j=h,m,i. Importer also incur a tariff cost $\tau$, so marginal cost is $c_i+\tau$ for importer. I further assume that multinational use exactly same technology and labor type both in foreign country and home country. Hence, the order is given as: $c_{soe}>c_d>c_m=c_i$ and after tariff imposed $c_{soe}>c_i+\tau>c_d>c_m$. Each company produced where marginal cost is equal to the marginal revenue. So by using demand equations in \ref{eq:demand} and profit maximizing condition, we have find prises. 
\end{frame}

\begin{frame}
\begin{equation}\label{eq:price}
\begin{split}
p_d&=\frac{\epsilon}{\epsilon-1}c_d\\
p_i&=\frac{\epsilon}{\epsilon-1}c_i(1+\tau)\\
p_m&=\frac{\epsilon}{\epsilon-1}c_m\\
p_{soe}&=\frac{\epsilon}{\epsilon-1}c_{soe}
\end{split}
\end{equation} 
The profit functions takes form
\begin{equation}\label{eq:profit}
\begin{split}
\pi_m&=(p_m-c_m)x_m\\
\pi_i&=(p_i-(c_i+\tau)x_i\\
\pi_d&=(p_d-c_d)x_d
\end{split}
\end{equation}
After substituting prices in \ref{eq:price} into the \ref{eq:profit}
we get 
\begin{equation}\label{eq:maxprofit}
\pi_j=\frac{p_jx_j}{\epsilon}, j=d,i,m,soe
\end{equation}
\end{frame}

\begin{frame}
\begin{equation}
G(n_m,\tau,t)\equiv\alpha V+\beta n_{soe} \pi_{soe}+ n_mt\pi_m+n_i\tau x_i+n_dt\pi_d,
\end{equation}
where $ \beta $ is the weight given SOE profit, E is employment level in SOE,
$ \tau $ is tariff and $ t $ is the profit tax applied to all private and multinational companies, $ \alpha $ is the weight assign to the welfare of the society and $V=Income+\frac{q^{1-\theta}}{\theta-1}$. Lets further assume that multinational companies pays a wage premium $w-1>0$  and to produce one unit multinational companies need $a_m$. Hence total wage premium paid by multinational is equal to the $(w-1)n_ma_mx_m$. The multinational price is $p_m=wa_m\epsilon/\epsilon-1$, then total wage premium can be written $(\epsilon-1)(w-1)/w\epsilon]n_mp_mx_m$. So utility takes form after including wage premium.
\begin{equation}
U=L+\frac{\epsilon-1}{\epsilon}\frac{w-1}{w}n_mp_mx_m+\frac{1}{\theta-1}[n_dp_dx_d+n_ip_ix_i+n_mp_mx_m+n_{soe}p_{soe}x_{soe}]
\end{equation}
\end{frame}

\begin{frame}
Hence government welfare maximization function is
\begin{equation}
\begin{split}
G(\tau,t,m)&\equiv\alpha V+\beta n_{soe} \pi_{soe}+ n_mt\pi_m+n_i\tau x_i+n_dt\pi_d\\
&=\alpha L + \alpha \frac{\epsilon-1}{\epsilon}\frac{w-1}{w}mp_mx_m\\
&+ \frac{\alpha}{\theta-1}[n_dp_dx_d+n_ip_ix_i+n_mp_mx_m+n_{soe}p_{soe}x_{soe}]\\
&+\beta n_{soe} \pi_{soe}+ n_mt\pi_m+n_i\tau x_i+n_dt\pi_d
\end{split}
\end{equation}
\end{frame}

\begin{frame}
	Hence government welfare maximization function is
	\begin{equation}
	\begin{split}
	s_{imt}&=-\beta s_{soe_{it}}+\alpha(\epsilon-1)(1-1/w_{it})-\epsilon\frac{\tau_t}{p_ft}s_{iit}+Complicated_term
	\end{split}
	\end{equation}
\end{frame}






\end{document}
